\documentclass[16pt]{ctexart} % Increase font size
\usepackage{xpinyin}
\usepackage{setspace} % Package to control spacing
\usepackage{reledmac} % three column footnotes
\usepackage[a4paper, total={7.5in, 10in}]{geometry}
\arrangementX[A]{threecol}
\let\footnote\footnoteA
\renewcommand{\baselinestretch}{2} % Adjust line spacing globally

% Manually adjust the font size to ensure it takes effect
\makeatletter
\renewcommand{\normalsize}{\@setfontsize{\normalsize}{16pt}{16pt}} % Define base font size
\makeatother

\begin{document}

\section*{Generated Graded Reader}

\xpinyin*{\textbf{王明} 和 \textbf{李华} 是好朋友。}
\xpinyin*{\textbf{王明} 住在 \underline{学校}\footnote{\label{1} school} 旁边。}
\xpinyin*{今天是星期六,他们去 \underline{公园}\footnote{\label{2} park} 玩。}
\xpinyin*{\textbf{李华} 喜欢 \underline{跑步}\footnote{\label{3} running},但是 \textbf{王明} 觉得 \underline{跑步}\footnotemark[3] 很累。}
\xpinyin*{他们坐在 \underline{椅子}\footnote{\label{4} chair} 上休息。}
\xpinyin*{\textbf{王明} 说:“我们去 \underline{商店}\footnote{\label{5} store} 吃点东西吧!”}
\xpinyin*{\textbf{李华} 说:“好啊,我想喝 \underline{牛奶}\footnote{\label{6} milk}。”}
\xpinyin*{他们去 \underline{商店}\footnotemark[5] 买了 \underline{牛奶}\footnotemark[6] 和 \underline{面包}\footnote{\label{7} bread}。}
\xpinyin*{吃完后,他们一起回家了。}


\xpinyin*{ 晚上,\textbf{王明} 在家 \underline{看书}\footnote{\label{8} reading}。}
\xpinyin*{他喜欢 \underline{历史}\footnote{\label{9} history} 书,也喜欢 \underline{小说}\footnote{\label{10} novels}。}
\xpinyin*{\textbf{王明} 的 \underline{妈妈}\footnote{\label{11} mother} 说:“别看太久,早点 \underline{睡觉}\footnote{\label{12} sleep}。”}
\xpinyin*{\textbf{王明} 说:“好的,妈妈!”}
\xpinyin*{第二天早上,\textbf{王明} 去 \underline{超市}\footnote{\label{13} supermarket} 买 \underline{鸡蛋}\footnote{\label{14} eggs} 和 \underline{蔬菜}\footnote{\label{15} vegetables}。}
\xpinyin*{他在 \underline{收银台}\footnote{\label{16} checkout counter} 付钱,然后回家做 \underline{早饭}\footnote{\label{17} breakfast}。}
\xpinyin*{\textbf{李华} 来了,他说:“你做了什么?”}
\xpinyin*{\textbf{王明} 说:“我做了 \underline{鸡蛋}\footnotemark[14] 和 \underline{米饭}\footnote{\label{18} rice}。”}
\xpinyin*{他们一起吃 \underline{早饭}\footnotemark[17],然后去 \underline{图书馆}\footnote{\label{19} library} 学习。}



\end{document}
