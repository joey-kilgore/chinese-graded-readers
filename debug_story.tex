\documentclass[16pt]{ctexart} % Increase font size
\usepackage{xpinyin}
\usepackage{setspace} % Package to control spacing
\usepackage{reledmac} % three column footnotes
\arrangementX[A]{threecol}
\let\footnote\footnoteA
\renewcommand{\baselinestretch}{2} % Adjust line spacing globally

% Manually adjust the font size to ensure it takes effect
\makeatletter
\renewcommand{\normalsize}{\@setfontsize{\normalsize}{16pt}{16pt}} % Define base font size
\makeatother

\newcommand{\underword}[2]{\underline{#1}\footnote{#2}}

\begin{document}

\section*{Generated Graded Reader}


\xpinyin*{\textbf{王明} 和 \textbf{李华} 是好朋友。}
\xpinyin*{\textbf{王明} 住在 \underword{学校}{school} 旁边。}
\xpinyin*{今天是星期六,他们去 \underword{公园}{park} 玩。}
\xpinyin*{\textbf{李华} 喜欢 \underword{跑步}{running},但是 \textbf{王明} 觉得 \underword{跑步}{running} 很累。}
\xpinyin*{他们坐在 \underword{椅子}{chair} 上休息。}
\xpinyin*{\textbf{王明} 说:“我们去 \underword{商店}{store} 吃点东西吧!”}
\xpinyin*{\textbf{李华} 说:“好啊,我想喝 \underword{牛奶}{milk}。”}
\xpinyin*{他们去 \underword{商店}{store} 买了 \underword{牛奶}{milk} 和 \underword{面包}{bread}。}
\xpinyin*{吃完后,他们一起回家了。}


\xpinyin*{ 晚上,\textbf{王明} 在家 \underword{看书}{reading}。}
\xpinyin*{他喜欢 \underword{历史}{history} 书,也喜欢 \underword{小说}{novels}。}
\xpinyin*{\textbf{王明} 的 \underword{妈妈}{mother} 说:“别看太久,早点 \underword{睡觉}{sleep}。”}
\xpinyin*{\textbf{王明} 说:“好的,妈妈!”}
\xpinyin*{第二天早上,\textbf{王明} 去 \underword{超市}{supermarket} 买 \underword{鸡蛋}{eggs} 和 \underword{蔬菜}{vegetables}。}
\xpinyin*{他在 \underword{收银台}{checkout counter} 付钱,然后回家做 \underword{早饭}{breakfast}。}
\xpinyin*{\textbf{李华} 来了,他说:“你做了什么?”}
\xpinyin*{\textbf{王明} 说:“我做了 \underword{鸡蛋}{eggs} 和 \underword{米饭}{rice}。”}
\xpinyin*{他们一起吃 \underword{早饭}{breakfast},然后去 \underword{图书馆}{library} 学习。}



\end{document}
